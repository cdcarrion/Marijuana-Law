\documentclass[11pt,]{article}
\usepackage[left=1in,top=1in,right=1in,bottom=1in]{geometry}
\newcommand*{\authorfont}{\fontfamily{phv}\selectfont}
\usepackage[sc, osf]{mathpazo}


  \usepackage[T1]{fontenc}
  \usepackage[utf8]{inputenc}



\usepackage{abstract}
\renewcommand{\abstractname}{}    % clear the title
\renewcommand{\absnamepos}{empty} % originally center

\renewenvironment{abstract}
 {{%
    \setlength{\leftmargin}{0mm}
    \setlength{\rightmargin}{\leftmargin}%
  }%
  \relax}
 {\endlist}

\makeatletter
\def\@maketitle{%
  \newpage
%  \null
%  \vskip 2em%
%  \begin{center}%
  \let \footnote \thanks
    {\fontsize{18}{20}\selectfont\raggedright  \setlength{\parindent}{0pt} \@title \par}%
}
%\fi
\makeatother


\renewcommand*\thetable{A.\arabic{table}}
\renewcommand*\thefigure{A.\arabic{figure}}


\setcounter{secnumdepth}{0}

\usepackage{longtable,booktabs}

\usepackage{graphicx,grffile}
\makeatletter
\def\maxwidth{\ifdim\Gin@nat@width>\linewidth\linewidth\else\Gin@nat@width\fi}
\def\maxheight{\ifdim\Gin@nat@height>\textheight\textheight\else\Gin@nat@height\fi}
\makeatother
% Scale images if necessary, so that they will not overflow the page
% margins by default, and it is still possible to overwrite the defaults
% using explicit options in \includegraphics[width, height, ...]{}
\setkeys{Gin}{width=\maxwidth,height=\maxheight,keepaspectratio}

\title{El Efecto de la Legalización de la Marihuana sobre la Criminalidad en
los Estados Unidos: Política Económica \thanks{La replicacion de los archivos están disponibles en la página Github del
autor (\href{https://github.com/cristian1512}{github.com/cristian1512})}  }



\author{\Large Cristian Carrión\vspace{0.05in} \newline\normalsize\emph{Escuela Politécnica Nacional}  }


\date{}

\usepackage{titlesec}

\titleformat*{\section}{\normalsize\bfseries}
\titleformat*{\subsection}{\normalsize\itshape}
\titleformat*{\subsubsection}{\normalsize\itshape}
\titleformat*{\paragraph}{\normalsize\itshape}
\titleformat*{\subparagraph}{\normalsize\itshape}





\newtheorem{hypothesis}{Hypothesis}
\usepackage{setspace}

\makeatletter
\@ifpackageloaded{hyperref}{}{%
\ifxetex
  \PassOptionsToPackage{hyphens}{url}\usepackage[setpagesize=false, % page size defined by xetex
              unicode=false, % unicode breaks when used with xetex
              xetex]{hyperref}
\else
  \PassOptionsToPackage{hyphens}{url}\usepackage[unicode=true]{hyperref}
\fi
}

\@ifpackageloaded{color}{
    \PassOptionsToPackage{usenames,dvipsnames}{color}
}{%
    \usepackage[usenames,dvipsnames]{color}
}
\makeatother
\hypersetup{breaklinks=true,
            bookmarks=true,
            pdfauthor={Cristian Carrión (Escuela Politécnica Nacional)},
             pdfkeywords = {Cannabis, marihuana, drogas ilegales, crimen.},  
            pdftitle={El Efecto de la Legalización de la Marihuana sobre la Criminalidad en
los Estados Unidos: Política Económica},
            colorlinks=true,
            citecolor=blue,
            urlcolor=blue,
            linkcolor=magenta,
            pdfborder={0 0 0}}
\urlstyle{same}  % don't use monospace font for urls

% set default figure placement to htbp
\makeatletter
\def\fps@figure{htbp}
\makeatother

\usepackage[spanish]{babel}
\usepackage{caption}


% add tightlist ----------
\providecommand{\tightlist}{%
\setlength{\itemsep}{0pt}\setlength{\parskip}{0pt}}

\begin{document}
	
% \pagenumbering{arabic}% resets `page` counter to 1 
%%\renewcommand*{\thepage}{A--\arabic{page}}
%
% \maketitle

{% \usefont{T1}{pnc}{m}{n}
\setlength{\parindent}{0pt}
\thispagestyle{plain}
{\fontsize{18}{20}\selectfont\raggedright 
\maketitle  % title \par  

}

{
   \vskip 13.5pt\relax \normalsize\fontsize{11}{12} 
\textbf{\authorfont Cristian Carrión} \hskip 15pt \emph{\small Escuela Politécnica Nacional}   

}

}








\begin{abstract}

    \hbox{\vrule height .2pt width 39.14pc}

    \vskip 8.5pt % \small 

\noindent El debate ha rodeado la legalización de la marihuana con fines médicos o
recreativos durante décadas, algunos han argumentado que la legalización
de la marihuana medicinal representa una amenaza para la salud pública y
la seguridad. En los últimos años, algunos estados de EE.UU. han
legalizado la marihuana con fines recreativos, reactivando el interés
político y público en el impacto de la legalización de la marihuana en
una serie de resultados.


\vskip 8.5pt \noindent \emph{Keywords}: Cannabis, marihuana, drogas ilegales, crimen. \par

    \hbox{\vrule height .2pt width 39.14pc}



\end{abstract}


\vskip 6.5pt

{
\hypersetup{linkcolor=black}
\setcounter{tocdepth}{2}
\tableofcontents
}

\noindent  \hypertarget{introduccion}{%
\section{Introducción}\label{introduccion}}

La legalización de la marihuana para uso recreativo en los Estados
Unidos sigue siendo un tema muy debatido a medida que más estados
legalizan la marihuana para uso recreativo y con fines médicos. El tema
abordado en este trabajo es si la \emph{Legalización de la Marihuana
(LM)} tiene el efecto de aumentar el crimen, si bien hay muchos
mecanismos por los cuales el LM podría afectar los índices de
delincuencia, como se mencionará más adelante, los comentarios que
apoyan la legalización de la marihuana se centran en una posible
disminución de la delincuencia debido a la reducción en el mercado y la
actividad delictiva asociada con ella(Maier, Mannes, and Koppenhofer
2017). Si la marihuana se legaliza o incluso se despenaliza, se
argumenta que los agentes de la ley dedicarán menos tiempo y recursos a
hacer cumplir las leyes(Caulkins and Kilmer 2016). Por otra parte, los
defensores de la LM comentan sobre los beneficios de salud para las
personas con ciertas enfermedades y afecciones médicas(Eddy 2005).
Uruguay se convirtió en el primer país del mundo en legalizar
completamente la marihuana se enfocará este análisis en los Estados
Unidos porque ha habido muchos cambios legales con respecto a la
despenalización o legalización de la marihuana en los últimos años, la
tendencia de la criminalidad neta se puede apreciar en la Figura
\ref{fig:plot1}.

\emph{El propósito de este análisis de datos no es explorar si existe
una relación entre la legalización y despenalización de la marihuana y
su uso, sino más bien observar la relación, si existe, entre las leyes
de la marihuana y las tasas de delincuencia.}

\begin{figure}
\centering
\includegraphics{mar_doc_files/figure-latex/graph1-1.pdf}
\caption{\label{fig:plot1} Criminalidad Neta en los EE.UU.}
\end{figure}

\hypertarget{metodos}{%
\section{Métodos}\label{metodos}}

\hypertarget{datos-y-medidas}{%
\subsection{Datos y Medidas}\label{datos-y-medidas}}

\hypertarget{variable-dependiente}{%
\subsubsection{Variable Dependiente}\label{variable-dependiente}}

Los datos sobre los cinco delitos (Asesinato y homicidio no negligente,
Violación, Robo y Asalto agravado) entre 1980 y 2014 se obtuvieron del
\emph{FBI's Uniform Crime Reporting} \href{https://bjs.gov/}{(UCR)}.
Todos los datos se recopilaron para cada uno de los 50 estados de EE.UU.
A lo largo del período de tiempo de 24 años para un total de N = 1785.

\hypertarget{variable-independiente}{%
\subsubsection{Variable Independiente}\label{variable-independiente}}

Para determinar si y cuándo ocurrió la LM dentro de un estado, buscamos
en el sitio web legislativo oficial para cada estado de los EE. UU.
Entre 1980 y 2014, los siguientes 23 estados legalizaron la marihuana
para uso médico, con el año en que se aprobó como se observa en la
Cuadro A.1. El año de inicio de la LM se obtuvo del sitio web oficial
\href{https://norml.org/states}{NORML} para cada estado(NORML n.d.). La
variable Dummy representa el número de años que la ley ha estado vigente
con un valor de cero para todos los años anteriores a la aprobación de
la ley, un valor de 1 para los años en que se aprobó la ley para
capturar cualquier cambio en el Tendencia lineal del delito que se puede
observar a lo largo del tiempo para corroborar los opositores de la LM
si están en lo cierto(Morris et al. 2014).

\hypertarget{variables-de-control}{%
\subsubsection{Variables de Control}\label{variables-de-control}}

Las variables sociodemográficas se incluyeron en el análisis para ayudar
a controlar una amplia gama de otras influencias que varían en el
tiempo. Específicamente, incluyen:

\begin{itemize}
\tightlist
\item
  El porcentaje de la fuerza laboral civil desempleada de cada estado,
  se obtuvo del sitio web de la Oficina de Estadísticas Laborales
  (\href{https://www.bls.gov/lau/}{BLS})
\item
  La tasa de empleo total, se obtuvo del sitio web de la Oficina de
  Estadísticas Laborales (\href{https://www.bls.gov/sae/}{BLS})
\item
  El porcentaje de la población que vive por debajo del umbral de
  pobreza, se obtuvo de la
  \href{https://www.census.gov/topics/income-poverty/poverty.html}{Oficina
  del Censo}
\item
  La tasa de consumo de cerveza per cápita(Scribner et al. 1999), los
  datos sobre el consumo de cerveza se tomaron del sitio web del
  National Institute on Alcohol Abuse and Alcoholism
  (\href{https://pubs.niaaa.nih.gov/publications/surveillance110/CONS16.htm}{(NIAAA)})
\end{itemize}

Las estadísticas de resumen para estas variables explicativas se
presentan en el Cuadro \ref{tab:descript}.

\begin{longtable}[]{@{}llll@{}}
\caption{Información estatal sobre legalización y
Despenalización}\tabularnewline
\toprule
\begin{minipage}[b]{0.20\columnwidth}\raggedright
\textbf{Legalizado} \emph{(4 Estados)}\strut
\end{minipage} & \begin{minipage}[b]{0.26\columnwidth}\raggedright
\textbf{Marihuana Medicinal Legalizado}\strut
\end{minipage} & \begin{minipage}[b]{0.24\columnwidth}\raggedright
\textbf{Despenalizado} \emph{(18 Estados)}\strut
\end{minipage} & \begin{minipage}[b]{0.19\columnwidth}\raggedright
\textbf{Ilegal} \emph{(22 Estados)}\strut
\end{minipage}\tabularnewline
\midrule
\endfirsthead
\toprule
\begin{minipage}[b]{0.20\columnwidth}\raggedright
\textbf{Legalizado} \emph{(4 Estados)}\strut
\end{minipage} & \begin{minipage}[b]{0.26\columnwidth}\raggedright
\textbf{Marihuana Medicinal Legalizado}\strut
\end{minipage} & \begin{minipage}[b]{0.24\columnwidth}\raggedright
\textbf{Despenalizado} \emph{(18 Estados)}\strut
\end{minipage} & \begin{minipage}[b]{0.19\columnwidth}\raggedright
\textbf{Ilegal} \emph{(22 Estados)}\strut
\end{minipage}\tabularnewline
\midrule
\endhead
\begin{minipage}[t]{0.20\columnwidth}\raggedright
Alaska (2014)\strut
\end{minipage} & \begin{minipage}[t]{0.26\columnwidth}\raggedright
Alaska(1998)\strut
\end{minipage} & \begin{minipage}[t]{0.24\columnwidth}\raggedright
California\strut
\end{minipage} & \begin{minipage}[t]{0.19\columnwidth}\raggedright
Alabama\strut
\end{minipage}\tabularnewline
\begin{minipage}[t]{0.20\columnwidth}\raggedright
Colorado (2012)\strut
\end{minipage} & \begin{minipage}[t]{0.26\columnwidth}\raggedright
Arizona (2011)\strut
\end{minipage} & \begin{minipage}[t]{0.24\columnwidth}\raggedright
Colorado\strut
\end{minipage} & \begin{minipage}[t]{0.19\columnwidth}\raggedright
Arkansas\strut
\end{minipage}\tabularnewline
\begin{minipage}[t]{0.20\columnwidth}\raggedright
Oregon (2014)\strut
\end{minipage} & \begin{minipage}[t]{0.26\columnwidth}\raggedright
California (1996)\strut
\end{minipage} & \begin{minipage}[t]{0.24\columnwidth}\raggedright
Connecticut\strut
\end{minipage} & \begin{minipage}[t]{0.19\columnwidth}\raggedright
Florida\strut
\end{minipage}\tabularnewline
\begin{minipage}[t]{0.20\columnwidth}\raggedright
Washington (2014)\strut
\end{minipage} & \begin{minipage}[t]{0.26\columnwidth}\raggedright
Colorado (2001)\strut
\end{minipage} & \begin{minipage}[t]{0.24\columnwidth}\raggedright
Delaware\strut
\end{minipage} & \begin{minipage}[t]{0.19\columnwidth}\raggedright
Georgia\strut
\end{minipage}\tabularnewline
\begin{minipage}[t]{0.20\columnwidth}\raggedright
\strut
\end{minipage} & \begin{minipage}[t]{0.26\columnwidth}\raggedright
Connecticut (2012)\strut
\end{minipage} & \begin{minipage}[t]{0.24\columnwidth}\raggedright
Maine\strut
\end{minipage} & \begin{minipage}[t]{0.19\columnwidth}\raggedright
Idaho\strut
\end{minipage}\tabularnewline
\begin{minipage}[t]{0.20\columnwidth}\raggedright
\strut
\end{minipage} & \begin{minipage}[t]{0.26\columnwidth}\raggedright
Delaware (2011)\strut
\end{minipage} & \begin{minipage}[t]{0.24\columnwidth}\raggedright
Maryland\strut
\end{minipage} & \begin{minipage}[t]{0.19\columnwidth}\raggedright
Indiana\strut
\end{minipage}\tabularnewline
\begin{minipage}[t]{0.20\columnwidth}\raggedright
\strut
\end{minipage} & \begin{minipage}[t]{0.26\columnwidth}\raggedright
Hawaii (2000)\strut
\end{minipage} & \begin{minipage}[t]{0.24\columnwidth}\raggedright
Massachusetts\strut
\end{minipage} & \begin{minipage}[t]{0.19\columnwidth}\raggedright
Iowa\strut
\end{minipage}\tabularnewline
\begin{minipage}[t]{0.20\columnwidth}\raggedright
\strut
\end{minipage} & \begin{minipage}[t]{0.26\columnwidth}\raggedright
Illinois (2013)\strut
\end{minipage} & \begin{minipage}[t]{0.24\columnwidth}\raggedright
Minnesota\strut
\end{minipage} & \begin{minipage}[t]{0.19\columnwidth}\raggedright
Kansas\strut
\end{minipage}\tabularnewline
\begin{minipage}[t]{0.20\columnwidth}\raggedright
\strut
\end{minipage} & \begin{minipage}[t]{0.26\columnwidth}\raggedright
Maine (1999)\strut
\end{minipage} & \begin{minipage}[t]{0.24\columnwidth}\raggedright
Mississippi\strut
\end{minipage} & \begin{minipage}[t]{0.19\columnwidth}\raggedright
Kentucky\strut
\end{minipage}\tabularnewline
\begin{minipage}[t]{0.20\columnwidth}\raggedright
\strut
\end{minipage} & \begin{minipage}[t]{0.26\columnwidth}\raggedright
Maryland (2014)\strut
\end{minipage} & \begin{minipage}[t]{0.24\columnwidth}\raggedright
Missouri\strut
\end{minipage} & \begin{minipage}[t]{0.19\columnwidth}\raggedright
Louisiana\strut
\end{minipage}\tabularnewline
\begin{minipage}[t]{0.20\columnwidth}\raggedright
\strut
\end{minipage} & \begin{minipage}[t]{0.26\columnwidth}\raggedright
Massachusetts (2013)\strut
\end{minipage} & \begin{minipage}[t]{0.24\columnwidth}\raggedright
Nebraska\strut
\end{minipage} & \begin{minipage}[t]{0.19\columnwidth}\raggedright
North Dakota\strut
\end{minipage}\tabularnewline
\begin{minipage}[t]{0.20\columnwidth}\raggedright
\strut
\end{minipage} & \begin{minipage}[t]{0.26\columnwidth}\raggedright
Michigan (2008)\strut
\end{minipage} & \begin{minipage}[t]{0.24\columnwidth}\raggedright
Nevada\strut
\end{minipage} & \begin{minipage}[t]{0.19\columnwidth}\raggedright
Oklahoma\strut
\end{minipage}\tabularnewline
\begin{minipage}[t]{0.20\columnwidth}\raggedright
\strut
\end{minipage} & \begin{minipage}[t]{0.26\columnwidth}\raggedright
Minnesota (2014)\strut
\end{minipage} & \begin{minipage}[t]{0.24\columnwidth}\raggedright
New York\strut
\end{minipage} & \begin{minipage}[t]{0.19\columnwidth}\raggedright
Pennsylvania\strut
\end{minipage}\tabularnewline
\begin{minipage}[t]{0.20\columnwidth}\raggedright
\strut
\end{minipage} & \begin{minipage}[t]{0.26\columnwidth}\raggedright
Montana (2004)\strut
\end{minipage} & \begin{minipage}[t]{0.24\columnwidth}\raggedright
North Carolina\strut
\end{minipage} & \begin{minipage}[t]{0.19\columnwidth}\raggedright
South Carolina\strut
\end{minipage}\tabularnewline
\begin{minipage}[t]{0.20\columnwidth}\raggedright
\strut
\end{minipage} & \begin{minipage}[t]{0.26\columnwidth}\raggedright
Nevada (2001)\strut
\end{minipage} & \begin{minipage}[t]{0.24\columnwidth}\raggedright
Ohio\strut
\end{minipage} & \begin{minipage}[t]{0.19\columnwidth}\raggedright
South Dakota\strut
\end{minipage}\tabularnewline
\begin{minipage}[t]{0.20\columnwidth}\raggedright
\strut
\end{minipage} & \begin{minipage}[t]{0.26\columnwidth}\raggedright
New Hampshire (2013)\strut
\end{minipage} & \begin{minipage}[t]{0.24\columnwidth}\raggedright
Oregon\strut
\end{minipage} & \begin{minipage}[t]{0.19\columnwidth}\raggedright
Tennessee\strut
\end{minipage}\tabularnewline
\begin{minipage}[t]{0.20\columnwidth}\raggedright
\strut
\end{minipage} & \begin{minipage}[t]{0.26\columnwidth}\raggedright
New Jersey (2010)\strut
\end{minipage} & \begin{minipage}[t]{0.24\columnwidth}\raggedright
Rhode Island\strut
\end{minipage} & \begin{minipage}[t]{0.19\columnwidth}\raggedright
Texas\strut
\end{minipage}\tabularnewline
\begin{minipage}[t]{0.20\columnwidth}\raggedright
\strut
\end{minipage} & \begin{minipage}[t]{0.26\columnwidth}\raggedright
New Mexico (2007)\strut
\end{minipage} & \begin{minipage}[t]{0.24\columnwidth}\raggedright
Vermont\strut
\end{minipage} & \begin{minipage}[t]{0.19\columnwidth}\raggedright
Utah\strut
\end{minipage}\tabularnewline
\begin{minipage}[t]{0.20\columnwidth}\raggedright
\strut
\end{minipage} & \begin{minipage}[t]{0.26\columnwidth}\raggedright
New York (2014)\strut
\end{minipage} & \begin{minipage}[t]{0.24\columnwidth}\raggedright
\strut
\end{minipage} & \begin{minipage}[t]{0.19\columnwidth}\raggedright
Virginia\strut
\end{minipage}\tabularnewline
\begin{minipage}[t]{0.20\columnwidth}\raggedright
\strut
\end{minipage} & \begin{minipage}[t]{0.26\columnwidth}\raggedright
Oregon (1998)\strut
\end{minipage} & \begin{minipage}[t]{0.24\columnwidth}\raggedright
\strut
\end{minipage} & \begin{minipage}[t]{0.19\columnwidth}\raggedright
West Virginia\strut
\end{minipage}\tabularnewline
\begin{minipage}[t]{0.20\columnwidth}\raggedright
\strut
\end{minipage} & \begin{minipage}[t]{0.26\columnwidth}\raggedright
Rhode Island (2005)\strut
\end{minipage} & \begin{minipage}[t]{0.24\columnwidth}\raggedright
\strut
\end{minipage} & \begin{minipage}[t]{0.19\columnwidth}\raggedright
Wisconsin\strut
\end{minipage}\tabularnewline
\begin{minipage}[t]{0.20\columnwidth}\raggedright
\strut
\end{minipage} & \begin{minipage}[t]{0.26\columnwidth}\raggedright
Vermont (2004)\strut
\end{minipage} & \begin{minipage}[t]{0.24\columnwidth}\raggedright
\strut
\end{minipage} & \begin{minipage}[t]{0.19\columnwidth}\raggedright
Wyoming\strut
\end{minipage}\tabularnewline
\begin{minipage}[t]{0.20\columnwidth}\raggedright
\strut
\end{minipage} & \begin{minipage}[t]{0.26\columnwidth}\raggedright
Washington (1998)\strut
\end{minipage} & \begin{minipage}[t]{0.24\columnwidth}\raggedright
\strut
\end{minipage} & \begin{minipage}[t]{0.19\columnwidth}\raggedright
\strut
\end{minipage}\tabularnewline
\bottomrule
\end{longtable}

\begin{table}[!htbp] \centering 
  \caption{Estadística Descriptiva para las Variables Usadas en el Análisis} 
  \label{tab:descript} 
\small 
\begin{tabular}{@{\extracolsep{5pt}}lccccc} 
\\[-1.8ex]\hline \\[-1.8ex] 
Statistic & \multicolumn{1}{c}{N} & \multicolumn{1}{c}{Mean} & \multicolumn{1}{c}{St. Dev.} & \multicolumn{1}{c}{Min} & \multicolumn{1}{c}{Max} \\ 
\hline 
\hline
{\bf Variables Independientes (sin log)} & & & & & \\
\hline \\[-1.8ex] 
Asesinato & 1,785 & 366.68 & 513.68 & 1 & 4,096 \\ 
Violación & 1,785 & 1,799.21 & 2,088.69 & 50 & 13,693 \\ 
Robo & 1,785 & 9,588.52 & 16,644.69 & 41 & 130,897 \\ 
Asalto Agravado & 1,785 & 17,160.13 & 24,674.18 & 213 & 198,045 \\ 
\hline
{\bf Variables Dependientes} & & & & & \\
\hline
Tasa de Desempleo & 1,100 & 13.05 & 3.38 & 5.60 & 24.60 \\ 
Tasa de pobreza & 1,750 & 25.28 & 27.27 & 1.69 & 173.11 \\ 
Consumo de Cerveza & 1,750 & 54.27 & 58.34 & 4.66 & 314.93 \\ 
Ley de Marihuana & 1,785 & 0.11 & 0.32 & 0 & 1 \\ 
\hline 
\hline \\[-1.8ex] 
\multicolumn{6}{l}{\textit{Nota:} Las estadísticas descriptivas corresponden al periodo 1980-2014} \\ 
\end{tabular} 
\end{table}

\hypertarget{plan-de-analisis}{%
\subsection{Plan de Análisis}\label{plan-de-analisis}}

Para identificar el efecto de la LM en el crimen, usamos un diseño de
panel de efectos aleatorios, que explota la variación dentro del estado
introducida por la LM en 50 estados durante el período de observación de
22 años con la depuración de los datos perdidos como se observa en el
Cuadro \ref{tab:misst}. Esto permite evaluar si los estados que
adoptaron la LM experimentaron cambios en la tendencia de la
delincuencia al analizar las tasas de delincuencia a lo largo del tiempo
y comparar esos cambios con las tendencias de la tasa de criminalidad
entre los estados que no aprobaron la LM. Además, también incluimos
``efectos aleatorios por año'', que capturan cualquier influencia
nacional sobre el crimen que no se refleja en ninguna de las variables
explicativas que varían con el tiempo. Los errores estándar robustos se
agrupan a nivel estatal para evitar errores estándar sesgados debido a
la no independencia de los puntos de datos a lo largo del tiempo(Morris
et al. 2014).

\begin{table}[!htbp] \centering 
  \caption{Porcentaje de los datos faltantes antes de la imputación} 
  \label{tab:misst} 
\begin{tabular}{@{\extracolsep{5pt}} lc} 
\\[-1.8ex]\hline 
\hline \\[-1.8ex] 
\textbf{Variables} & \textbf{\% Faltantes} \\ 
\hline \\[-1.8ex] 
Homicidio & 0.00\% \\ 
Violación & 0.00\% \\ 
Robo & 0.00\% \\ 
Asalto Agravado & 0.00\% \\ 
Tasa de Pobreza & 38.38\% \\ 
Tasa de Empleo & 1.96\% \\ 
Tasa de Desempleo & 1.96\% \\ 
Galones de cerbeza per capita & 1.96\% \\ 
Post-Ley & 0.00\% \\ 
\hline \\[-1.8ex] 
\end{tabular} 
\end{table}

\newpage

\hypertarget{resultados}{%
\section{Resultados}\label{resultados}}

Antes de consultar los resultados de los modelos de regresión, se
generaron una serie de regresiones discontinuas para el crimen total de
los 2 estados más grandes de EE.UU. que son California
(Legalizado-1996), Texas(Ileagal) y New Jersey el segundo estado más
peligroso(CNN n.d.) como se observa en la Figura \ref{plot2}. Tomar en
cuenta que existe 2 tendencias para el índice de crimen. Una tendencia
\emph{(lado izquierdo)} muestra la criminalidad por año, cuando todavía
no habían aprobado la LM que se tomará como base al estado de
California. Por lo tanto, el impacto de la LM contribuyen la recta
\emph{(lado derecho)} hasta el año vigente de aprobación. Como se
esperaba de la tendencia general de delitos durante este período, la
recta del lado derecho revela estos estados experimentaron una reducción
de crimen gradualmente para los estados de California y New Jersey para
el periodo 1980 a 2014. Los estados que aprueban la LM experimentaron
reducciones en el crimen. Estos \emph{RESULTADOS PRELIMINARES} sugieren
que la LM puede tener un efecto de reducción del crimen, pero hay que
tomar en cuenta que otros factores pueden estar relacionados con las
tendencias de las series de tiempo que no se han tomado en cuenta para
estas regresiones.

\begin{figure}
\centering
\includegraphics[width=0.8\textwidth,height=\textheight]{C:/Users/Usuario/Documents/Doc_espan/Marihuana_law/pol_imp3.PNG}
\caption{Regresión Discontinua de la tasa de Criminalidad en función de
los años \label{plot2}}
\end{figure}

Los resultados de los análisis de efectos aleatorios se presentan en el
Cuadro \ref{tab:mdl} Es importante tener en cuenta que se realizó una
prueba de Hausman para determinar si el modelo de efectos aleatorios era
preferible al modelo de efectos fijos. Los resultados clave obtenidos de
los análisis de efectos aleatorios se presentan en la fila 1 de la Tabla
\ref{tab:mdl} que se puede visualizar de mejor manera en la Figura
\ref{fig:plot3}, que revela el impacto de la variable de tendencia MML
en las tasas de criminalidad, mientras controla las otras variables
explicativas que varían en el tiempo. De los diferentes análisis de
regresión de efectos aleatorios surgieron hallazgos dignos de mención.

\begin{table}[!htbp] \centering 
  \caption{El impacto de la Ley de la Marihuana sobre la Criminalidad} 
  \label{tab:mdl} 
\small 
\begin{tabular}{@{\extracolsep{5pt}}lcccc} 
\\[-1.8ex]\hline \\[-1.8ex] 
\\[-1.8ex] &  &  &  & \emph{Asalto} \\
\\[-1.8ex] & \emph{Asesinato} & \emph{Violación} & \emph{Robo} & \emph{Agravado}\\ 
\\[-1.8ex] & \textbf{Modelo 1} & \textbf{Modelo 2} & \textbf{Modelo 3} & \textbf{Modelo 4}\\ 
\hline \\[-1.8ex] 
 Tasa de Pobreza & $-$0.024$^{***}$ & 0.004 & $-$0.013$^{*}$ & $-$0.008 \\ 
  & (0.008) & (0.006) & (0.007) & (0.008) \\ 
  Tasa de Empleo & $-$0.016$^{***}$ & $-$0.017$^{***}$ & $-$0.026$^{***}$ & $-$0.027$^{***}$ \\ 
  & (0.004) & (0.003) & (0.003) & (0.004) \\ 
  Tasa de Desemp & 0.001 & $-$0.025$^{***}$ & $-$0.008 & $-$0.044$^{***}$ \\ 
  & (0.008) & (0.006) & (0.007) & (0.008) \\ 
  Consumo de Cerveza & 0.012$^{***}$ & 0.012$^{***}$ & 0.016$^{***}$ & 0.016$^{***}$ \\ 
  & (0.002) & (0.002) & (0.002) & (0.002) \\ 
  Legalización Marihuana & $-$0.009$^{***}$ & 0.005$^{**}$ & 0.005$^{**}$ & 0.007$^{**}$ \\ 
  & (0.003) & (0.002) & (0.002) & (0.003) \\ 
\hline
{\bf Efecto Aleatorio} & & & &  \\
\hline
\# de Estados & 50 & 50 & 50 & 50 \\
Desv. Estand. Estados &  1.252 & 0.855 & 1.591 & 1.255 \\
\\
\# de Años & 22 & 22 & 22 & 22 \\
Desv. Estand. Años &  0.104 & 0.046 & 0.087 & 0.072 \\
\hline
 N & 1,100 & 1,100 & 1,100 & 1,100 \\ 
\hline \\[-1.8ex] 
\multicolumn{5}{l}{$^{***}$p $<$ .01; $^{**}$p $<$ .05; $^{*}$p $<$ .1} \\ 
\multicolumn{5}{l}{\textit{Nota:}  Las siguientes variables se dividieron por 100k: tasa de empleo y consumo de cerveza} \\ 
\end{tabular} 
\end{table}

\begin{figure}
\centering
\includegraphics{mar_doc_files/figure-latex/graph3-1.pdf}
\caption{\label{fig:plot3} Diagrama de los Coeficientes del Impacto de
la LM}
\end{figure}

\newpage

El impacto de la LM en el crimen fue negativo en uno de los modelos que
es de \emph{Homicidio}, lo que sugiere que la aprobación de la LM puede
tener un efecto no atenuante en el resto de delitos. Específicamente,
los resultados indican aproximadamente una reducción del 0.9 por ciento
en Asesinato, respectivamente, por cada año adicional en que la ley esté
vigente.

\hypertarget{conclusiones}{%
\section{Conclusiones}\label{conclusiones}}

Los efectos de la marihuana medicinal legalizada han sido muy debatidos
en los últimos años. Sin embargo, la investigación empírica sobre la
relación directa entre las leyes sobre la Marihuana y el crimen es
escasa y las consecuencias del consumo de marihuana en el crimen siguen
siendo desconocidas, por lo tanto, al final no se encontró que la LM
tenga un efecto de mejora del crimen para ninguno de los tipos de
delitos analizados.

Si bien es importante mantenerse cauteloso al interpretar estos
hallazgos como evidencia de que la LM reduce el crimen, estos resultados
se ajustan a la evidencia reciente y se ajustan a la idea de que la
legalización de la marihuana puede llevar a una reducción en el consumo
de alcohol debido a individuos que sustituyen a la marihuana por
alcohol. Además, los hallazgos actuales también deben tomarse en
contexto con la naturaleza de los datos disponibles. Se basan en los
registros oficiales de arresto (UCR), que no tienen en cuenta los
delitos que no se denunciaron a la policía. Por ende, esta evaluación de
impacto de la LM sobre las tasas de crimen no parecen tener ningun
efecto negativo en la criminalidad oficialmente reportada durante los
años en que las leyes están vigentes. También es importante tener en
cuenta que los datos de la UCR utilizados aquí no tuvieron en cuenta la
delincuencia juvenil, que puede o no estar vinculada empíricamente a la
LM de una forma u otra.

\newpage

\hypertarget{referencias}{%
\section{Referencias}\label{referencias}}

\setlength{\parindent}{-0.2in}
\setlength{\leftskip}{0.2in}
\setlength{\parskip}{8pt}
\vspace*{-0.2in}

\noindent

\hypertarget{refs}{}
\leavevmode\hypertarget{ref-Caulkins2016}{}%
Caulkins, Jonathan P., and Beau Kilmer. 2016. ``Considering marijuana
legalization carefully: insights for other jurisdictions from analysis
for Vermont.'' \emph{Addiction} 111 (12): 2082--9.
\url{https://doi.org/10.1111/add.13289}.

\leavevmode\hypertarget{ref-CNN}{}%
CNN. n.d. ``Las 10 ciudades más peligrosas de EE.UU.'' Accessed July 10,
2019.
\url{https://cnnespanol.cnn.com/2014/02/04/las-10-ciudades-mas-peligrosas-de-ee-uu/}.

\leavevmode\hypertarget{ref-Eddy2005}{}%
Eddy, Mark. 2005. ``Medical Marijuana: Review and Analysis of Federal
and State Policies,'' December.
\url{https://digital.library.unt.edu/ark:/67531/metacrs8244/}.

\leavevmode\hypertarget{ref-Maier2017}{}%
Maier, Shana L., Suzanne Mannes, and Emily L. Koppenhofer. 2017. ``The
Implications of Marijuana Decriminalization and Legalization on Crime in
the United States.'' \emph{Contemporary Drug Problems} 44 (2): 125--46.
\url{https://doi.org/10.1177/0091450917708790}.

\leavevmode\hypertarget{ref-Morris2014}{}%
Morris, Robert G., Michael TenEyck, J. C. Barnes, and Tomislav V.
Kovandzic. 2014. ``The effect of medical marijuana laws on crime:
Evidence from state panel data, 1990-2006.'' \emph{PLoS ONE} 9 (3).
\url{https://doi.org/10.1371/journal.pone.0092816}.

\leavevmode\hypertarget{ref-NORML}{}%
NORML. n.d. ``United States.'' Accessed July 6, 2019.
\url{https://norml.org/states}.

\leavevmode\hypertarget{ref-Scribner1999}{}%
Scribner, R, D Cohen, S Kaplan, and S H Allen. 1999. ``Alcohol
availability and homicide in New Orleans: conceptual considerations for
small area analysis of the effect of alcohol outlet density.''
\emph{Journal of Studies on Alcohol} 60 (3): 310--6.
\url{http://www.ncbi.nlm.nih.gov/pubmed/10371257}.




\newpage
\singlespacing 
\end{document}
